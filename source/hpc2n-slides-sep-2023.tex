\documentclass[usenames,dvipsnames]{beamer}

% ////////////////////////////////////////////////////////////////////////////
% Instructions for authors
% ////////////////////////////////////////////////////////////////////////////
% 
% Please scan the current document and notices how it is divided into sections
% Only one author per slide, please! This will reduce merge conflicts
% Create new commands in separate sections marked by your initials
% The document width is 78 characters.

% Configuration of beamer
\usetheme{Copenhagen}
\setbeamertemplate{navigation symbols}{}
\setbeamertemplate{footline}[frame number]

% Basic packages
\usepackage{graphics,graphicx,hyperref,amssymb,amsthm,color,multirow}
\usepackage{ragged2e,tablefootnote,bm}
\usepackage[normalem]{ulem}
\usepackage{hhline}
\usepackage{amsmath,subcaption,csquotes}
\usepackage[many]{tcolorbox}
\usepackage{listings}
\usepackage{hyperref}
\usepackage{verbatim}
\usepackage[]{textpos}
\usepackage{dirtytalk}
\usepackage{makecell}
\usepackage{arydshln}

\renewcommand{\emph}[1]{\textbf{#1}}
% Set the graphics path
% \graphicspath{{./graphics/people/}{./graphics/logos/}}

% ////////////////////////////////////////////////////////////////////////////
% LaTeX shortcut commands typically defined by CCKM
% ////////////////////////////////////////////////////////////////////////////

% Theorems
\newtheorem{algorithm}[theorem]{Algorithm}

% Finite dimensional vectorspaces
\newcommand{\R}{\mathbb{R}}
\newcommand{\Rn}{\R^n}
\newcommand{\Rp}{\R^p}
\newcommand{\Rnn}{\R^{n \times n}}
\newcommand{\Rnp}{\R^{n \times p}}
\newcommand{\myspan}{\text{span}}

% Equations
\newcommand{\be}{\begin{equation}}
  \newcommand{\ee}{\end{equation}}
\newcommand{\bes}{\begin{equation*}}
  \newcommand{\ees}{\end{equation*}}

% Matrices
\newcommand{\bbm}{\begin{bmatrix}}
  \newcommand{\ebm}{\end{bmatrix}}
\newcommand{\bpm}{\begin{pmatrix}}
  \newcommand{\epm}{\end{pmatrix}}

% Sets
\newcommand{\Ran}{\text{Ran}\:}
\newcommand{\Ker}{\text{Ker}\:}

% Floating point representation
\newcommand{\fl}{\text{fl}}

% ////////////////////////////////////////////////////////////////////////////
% Specialized graphics
% /////////////////////////////////////////////////////////////////////////// 

\usepackage{tikz}
\usetikzlibrary{shapes}
\setbeamerfont{block title}{size=\small}

\title[Introduction to HPC2N]{Introduction to HPC2N}
%\subtitle{}
%\author{Birgitte Bryds\o, Mirko Myllykoski, and Pedro Ojeda-May}
%\author{Birgitte Bryds\o, Pedro Ojeda-May}
\institute{HPC2N, Ume\aa{} University}

\titlegraphic{\center{\includegraphics[height=1.2cm]{figures/naiss.png}\hspace*{1.1cm}~
   \includegraphics[height=1.2cm]{figures/umu-logotyp-EN.png}\hspace*{1.1cm}~\includegraphics[height=1.2cm]{figures/hpc2n.png}
}}
%\date{21 September 2023}
\begin{document}

\frame{
  \maketitle
}

\frame{
\frametitle{HPC2N (HPC2N at a glance)}
\begin{itemize}
 \item \emph{High Performance Computing Center North (HPC2N)} is a competence center for Scientific and Parallel Computing
\end{itemize}
\begin{center}
 \includegraphics[width=0.4\textwidth]{figures/hpc2n-logo-text5.png}
\end{center}
\begin{itemize}
 \pause \item A part of \emph{National Academic Infrastructure for Super­computing in Sweden (NAISS)}
\end{itemize}
\begin{center}
 \includegraphics[height=15mm]{figures/naiss.png}
\end{center}
}

\frame{
\frametitle{HPC2N (HPC2N at a glance)}

Provides state-of-the-art resources and expertise:
\begin{itemize}
 \item Scalable and parallel \emph{HPC}
 \pause \item Large-scale \emph{storage facilities} (Project storage (Lustre), SweStore, Tape)
 \pause \item \emph{Grid and cloud} computing (WLCG NT1, Swedish Science Cloud)
\pause \item National Data Science Node in ”Epidemiology and
Biology of Infections” (DDLS)
 \pause \item Software for e-Science applications
 \pause \item All levels of user support
 \begin{itemize}
  \item Primary, advanced, dedicated
  \item Application Experts (AEs)
 \end{itemize}
\end{itemize}
}

\frame{
  \frametitle{HPC2N}

  \textbf{Primary objective:} to raise the national and local level of HPC competence and transfer HPC knowledge and technology to new users in academia and industry.
  }

  
\frame{
\frametitle{HPC2N (partners)}
\emph{HPC2N is hosted by} \\
\begin{center}\includegraphics[height=16mm]{figures/umu-logotyp-EN.png}\hspace{12mm} \\
  \end{center}
\emph{Partners}: \\
\begin{center}
  \includegraphics[height=1cm]{figures/irf.png}\hspace*{1.2cm}~
  \includegraphics[height=1cm]{figures/ltu.preview.png}\hspace*{1.2cm}~\includegraphics[height=1cm]{figures/mid_sweden_university.png}\hspace*{1.2cm}~\includegraphics[height=1cm]{figures/slu_new.png}
    \end{center}
}

\frame{
\frametitle{HPC2N (funding and collaborations)}
\begin{itemize}
	\item Funded mainly by \emph{Umeå University}, with contributions from the \emph{other HPC2N partners} \\
% \begin{center}
%  \vspace{5mm}
%  \includegraphics[height=18mm]{figures/umu-logotyp-EN.png}\hspace{10mm}
%  \includegraphics[height=10mm]{figures/SNIC_logo_lower.png}
%  \vspace{5mm}
% \end{center}
 \pause \item Involved in several \emph{projects and collaborations} \\
% \begin{itemize}
%  \item DDLS, EGI, EISCAT, eSSENCE, NOSEG, Swedish Science Cloud, \dots
 % \end{itemize}
 \begin{center}
  \includegraphics[height=1cm]{figures/essence.png}\hspace*{0.8cm}~
  \includegraphics[height=1cm]{figures/prace.png}\hspace*{0.8cm}~\includegraphics[height=1cm]{figures/algoryx.png}\\ \vspace{0.8cm} \includegraphics[height=1cm]{figures/WLCG-logo.png}\hspace*{0.8cm}~\includegraphics[height=1cm]{figures/eosc-nordic.png}\hspace*{0.8cm}~\includegraphics[height=1cm]{figures/eiscat-logo5.png}\\ \vspace{0.8cm} \includegraphics[height=1cm]{figures/ScilifeLab.png}\hspace*{0.8cm}~\includegraphics[height=1cm]{figures/skills4eosc.png}
    \end{center}
\end{itemize}
}

\frame{
\frametitle{HPC2N (training and other services)}
\begin{itemize}
 \item \emph{User support} (primary, advanced, dedicated)
 \begin{itemize}
  \item Research group meetings @ UmU
  \item Also at the partner sites
 \end{itemize}
 \pause \item \emph{User training and education program}
 \begin{itemize}
  \item 0.5 -- 3 days; ready-to-run exercises
  \item Introduction to HPC2N and Kebnekaise
  \item Parallel programming and tools (OpenMP, MPI, debugging, perf. analyzers, Matlab, R, MD simulation, ML, GPU, \dots)
    \begin{itemize}
      \item \textbf{Using Python in an HPC environment}, 1 December 2023
  \item \textbf{Introduction to Git}, 13-17 November 2023
  \item \textbf{Introduction to running R, Python, and Julia in HPC}, 17-19 October 2023
  \item \textbf{Workshop: Matlab in HPC}, 11, 18, 25/26 September 2023
  \item \textbf{Introduction to Kebnekaise}, 21 September 2023
\end{itemize}
 \end{itemize}
% \pause \item NGSSC / SeSE \& university courses
 \pause \item Workshops and seminars 
% \item<8-> Research \& Development --- Technology transfer
\end{itemize}
}

\frame{
\frametitle{HPC2N (personnel)}
\footnotesize
\begin{minipage}{0.48\textwidth}
\emph{Management}
\begin{itemize}
 \item Paolo Bientinesi, director
 \item Bj{\"o}rn Torkelsson, deputy director
 \item Lena Hellman, administrator
\end{itemize}
\pause \emph{Application experts}
\begin{itemize}
 \item Jerry Eriksson
% \item Mirko Myllykoski
 \item Pedro Ojeda May
\end{itemize}
\pause \emph{Others}
\begin{itemize}
% \item Bo K\r{a}gstr{\"o}m
 \item Mikael R{\"a}nnar (WLCG coord)
% \item Anders Backman
% \item Kenneth Bodin
% \item Claude Lacoursi\`{e}re (Algoryx)
 \item Research Engineers under DDLS, HPC2N/SciLifeLab
	 \begin{itemize}
		 \item System Developer, IT
		 \item Data Engineer
		 \item Data Steward 
	 \end{itemize}
\end{itemize}
\end{minipage}
\,
\begin{minipage}{0.48\textwidth}
\pause \emph{System and support}
\begin{itemize}
 \item Erik Andersson
 \item \emph{Birgitte Bryds{\"o}}
 \item Niklas Edmundsson (Tape coord)
 \item Ingemar F{\"a}llman
 \item Magnus Jonsson 
 \item Roger Oscarsson
 \item \emph{\r{A}ke Sandgren}
 \item Mattias Wadenstein (NeIC, Tier1)
 \item \emph{Lars Viklund}
\end{itemize}
\end{minipage}
}

\frame{
\frametitle{HPC2N (application experts)}
\begin{itemize}
 \item HPC2N provides advanced and dedicated support in the form of \emph{Application Experts (AEs)}:
\end{itemize}
\begin{tabular}{rp{7.3cm}}
 \pause \color{cyan} Jerry Eriksson & Profiling, Machine learning (DNN), MPI, OpenMP, OpenACC \\
% \pause \color{cyan} Mirko Myllykoski & General HPC, numerical linear algebra, GPUs (CUDA, OpenCL, ...), task-based parallelism \\
 \pause \color{cyan} Pedro Ojeda May & Molecular dynamics, Profiling, QM/MM, NAMD, Amber, Gromacs, GAUSSIAN, R \\
 \pause \color{cyan} \r{A}ke Sandgren & General high level programming assistance, VASP, Gromacs, Amber \\
\end{tabular}
\begin{itemize}
 \pause \item Contact through regular support 
 \begin{itemize}
  \item If you have a specific problem/question and/or need consultation 
 \end{itemize}
\end{itemize}
}

\frame{
\frametitle{HPC2N (users by discipline)}
\begin{itemize}
 \item Users from several scientific disciplines:
 \begin{itemize}
  \item Biosciences and medicine 
  \item Chemistry
  \item Computing science  
  \item Engineering 
  \item Materials science
  \item Mathematics and statistics 
  \item Physics including space physics
  \item ML, DL, and other AI
 \end{itemize}
\end{itemize}
}

\frame{
\frametitle{HPC2N (users by discipline, largest users)}
\begin{itemize}
 \item Users from several scientific disciplines:
 \begin{itemize}
  \item Biosciences and medicine 
  \item \emph{Chemistry}
  \item Computing science  
  \item Engineering 
  \item \emph{Materials science}
  \item Mathematics and statistics 
  \item \emph{Physics including space physics}
  \item {\color{cyan} Machine learning and artificial intelligence} (several new projects)
 \end{itemize}
\end{itemize}
}

%% \frame{
%% \frametitle{HPC2N (medium users by university)}
%% \begin{center}
%% \includegraphics[width=0.9\textwidth]{figures/medium-users.png}
%% Projects with allocations at HPC2N: 2014-01-01 to 2016-05-30
%% \end{center}
%% }

%% \frame{
%% \frametitle{HPC2N (large users by university)}
%% \begin{center}
%% \includegraphics[width=0.9\textwidth]{figures/large-users.png}
%% Projects with allocations at HPC2N: 2014-01-01 to 2016-05-30
%% \end{center}
%% }

\frame{
\frametitle{HPC2N (users by software)}
\begin{center}
\includegraphics[width=0.9\textwidth]{figures/software_v2.png}
\end{center}
}

\frame{
\frametitle{Kebnekaise}
\begin{itemize}
 \item \emph{The current supercomputer at HPC2N}
 \pause \item Named after a massif (contains some of Sweden's highest mountain peaks)
 \pause \item Kebnekaise was 
 \begin{itemize}
  \item delivered by Lenovo and 
  \item \emph{installed during the summer 2016}
 \end{itemize}
 \pause \item Opened up for general availability on November 7, 2016
 \pause \item In 2018, Kebnekaise was \emph{extended} with 
 \begin{itemize}
  \item 52 Intel Xeon Gold 6132 (Skylake) nodes, as well as 
  \item 10 NVidian V100 (Volta) GPU nodes
 \end{itemize}
 \pause \item In 2023, Kebnekaise was \emph{extended} with 
 \begin{itemize}
 \item 2 dual NVIDIA A100 GPU nodes
 \item one many-core AMD Zen3 CPU node
 \end{itemize}
\end{itemize}
}

\frame{
\frametitle{Kebnekaise (compute nodes)}
\begin{tabular}{rcp{6.0cm}}
 Name & \# & Description \\ \hline 
         \color{cyan} Compute-AMD Zen3 & 1 & \makecell[l]{AMD Zen3 (EPYC 7762), 2 x 64 cores, \\\emph{1 TB}, EDR Infiniband} \\ \hdashline
	\pause \color{cyan} Compute-skylake & \emph{52} & \makecell[l]{Intel Xeon Gold 6132, 2 x 14 cores, \\\emph{192 GB}, EDR Infiniband, \emph{AVX-512}} \\ \hdashline
        \pause \makecell[r]{\color{cyan} Compute \\ \small{\color{red} !!! Being phased out !!!}} & \emph{432} & \makecell[l]{Intel Xeon E5-2690v4, \emph{2 x 14 cores}, \\\emph{128 GB}, FDR Infiniband} \\ \hdashline
        \pause \color{cyan} Large Memory & 20 & \makecell[l]{Intel Xeon E7-8860v4, \emph{4 x 18 cores}, \\\emph{3072 GB}, EDR Infiniband} \\ \hdashline
% \pause \color{cyan} KNL & 36 & \makecell[l]{Intel \emph{Xeon Phi} 7250 (Knight's Landing), \\ 68 cores, 192 GB, 16 GB MCDRAM, \\FDR Infiniband}
\end{tabular}
}

 \frame{
 \frametitle{Kebnekaise (GPU nodes)}
 \begin{footnotesize}
 \begin{tabular}{rcp{7.0cm}}
 Name & \# & Description \\ \hline
 	\color{cyan} 2 x A100 & 2 & \makecell[l]{AMD Zen3 (AMD EPYC 7413), 2 x 24 cores, \\ 512 GB, EDR Infiniband, \\\emph{2 x NVidia A100}, \\2 x 6912 CUDA cores, \\ \emph{2 x 432 Tensor cores}} \\ \hdashline
        \pause \color{cyan} GPU-volta & 10 & \makecell[l]{Intel Xeon Gold 6132, 2 x 14 cores,\\ 192 GB, EDR Infiniband, \\\emph{2 x NVidia V100}, \\2 x 5120 CUDA cores, 2 x 16 GB VRAM, \\\emph{2 x 640 Tensor cores}} \\ \hdashline 
  \pause \makecell{\color{cyan} 4xGPU \\ \small{\color{red} !!! Being phased out !!!}} & 4 & \makecell[l]{Intel Xeon E5-2690v4, 2 x 14 cores,\\ 128 GB, FDR Infiniband,\\ \emph{4 x NVidia K80} \\ 8 x 2496 CUDA cores, 8 x 12 GB VRAM} \\ \hdashline
  \pause \makecell{\color{cyan} 2xGPU \\ \small{\color{red} !!! Being phased out !!!}} & 32 & \makecell[l]{Intel Xeon E5-2690v4, 2 x 14 cores,\\ 128 GB, FDR Infiniband,\\ \emph{2 x NVidia K80} \\ 4 x 2496 CUDA cores, 4 x 12 GB VRAM} \\ 
 \end{tabular}
 \end{footnotesize}
 }

\frame{
\frametitle{Kebnekaise (in numbers)}
\begin{itemize}
\item 553 nodes in 15 racks
  \pause \item Intel Broadwell and Skylake, AMD Zen3
  \pause \item NVidia A100, V100, K80 GPUs
% \begin{itemize}
%  \item 18840 available for users (the rest are for managing the cluster)
% \end{itemize}
 \pause \item More than \emph{135 TB memory}
 \pause \item 71 switches (Infiniband, Access and Management networks)
  \pause \item 16504 CPU cores
% \pause \item 728 TFlops/s Peak performance (expansion not included)
% \pause \item \emph{629 TFlops/s} Linpack (all parts, except expansion)
\pause \item 501760 CUDA cores
\pause \item 12800 Tensor cores
\end{itemize}
}

\frame{
\frametitle{Kebnekaise (HPC2N storage)}
\begin{itemize}
 \item Basically four types of storage are available at HPC2N:
 \begin{itemize}
  \pause \item {\color{cyan} Home directory}
  \begin{itemize}
   \item \texttt{/home/X/Xyz}, \texttt{\$HOME}, \texttt{$\sim$}
   \item 25 GB, user owned
  \end{itemize}
  \pause \item {\color{cyan} Project storage}
  \begin{itemize}
   \item \texttt{/proj/nobackup/abc}
   \item Shared among project members
  \end{itemize}
  \pause \item {\color{cyan} Local scratch space}
  \begin{itemize}
   \item \texttt{\$SNIC\_TMP}
   \item SSD (170GB), per job, per node, "volatile"
  \end{itemize}
  \pause \item {\color{cyan} Tape Storage}
  \begin{itemize}
   \item Backup
   \item \emph{Long term storage}
  \end{itemize}
 \end{itemize}
  \pause \item Also {\color{cyan} SweStore} --- disk based (dCache)
  \begin{itemize}
   \item Research Data Storage Infrastructure, for active research data and operated by NAISS, WLCG
  \end{itemize}
\end{itemize}
}

\frame{
\frametitle{Kebnekaise (projects)}
\begin{itemize}
 \item In order to use Kebnekaise, you must be a member of a \emph{compute project}
 \begin{itemize}
  \pause \item A compute project has a certain number of \emph{core hours} allocated for it per month
  \pause \item A regular CPU core cost 1 core hour per hour, other resources (e.g., GPUs) cost more
  \pause \item Not a hard limit but projects that go over the allocation get lower priority
 \end{itemize}
 \pause \item A compute project contains a certain amount of storage
 \begin{itemize}
  \item If more storage is required, you must be a member of a \emph{storage project}
 \end{itemize}
 \pause \item As Kebnekaise is a local cluster, you need to be affiliated with UmU, IRF, SLU, Miun, or LTU to use it
 \pause \item Projects are applied for through SUPR (https://supr.naiss.se) 
 % \pause \item Birgitte will cover more details
 %\pause \item I will cover more details in the next section, where we go more into detail about HPC2N and Kebnekaise. 
\end{itemize}
}

\end{document}






